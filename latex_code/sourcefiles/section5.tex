In our research project, using firm-level data from 2015 to 2017, we find that FDI has positive effects on labor market outcomes. Thus, we find that receiving FDI increases both firm-level employment and wages by 31.7\% and 25.6\%, respectively. Interestingly, analyzing whether the motive for the foreign investment matters for these effects does not reveal any significant differences. More specifically, the impact of FDI on employment and wages, respectively, does not vary by type of FDI when we distinguish between domestic market seeking FDI, export-oriented FDI, and technology-intensive FDI.  \\ \par

While in principle the propensity score matching techniques we employ allow us to identify the causal effect of FDI on employment and wages, the reliability of our estimates rely on the balancedness and common support assumptions to be fulfilled. As presented in Sections 3 and 4, we tried optimizing our models to find the best balance between the two criteria. However, it might be possible to further optimize our models in this respect. For example, overlap is quite poor in some regions when performing the analysis by type of FDI. Trimming the sample could be one way to address this issue – but, of course, this comes at the cost of losing observations which is why we opted against it in our analysis.  \\ \par 

In addition, our analysis is limited by data availability. There may be other factors driving FDI that, at the same time, also affect firm-level performance. For example, we do know the country of location of the firms in our dataset. Information on the location of the investor-origin and recipient country of FDI would enable us to control for both country-specific time-invariant characteristics by using country dummies, and time varying country variables such as macroeconomic conditions. Using growth variables relative to pre-treatment levels is another strategy to at least eliminate unobserved time-invariant confounders. We leave these optimizations to future ASP students.  