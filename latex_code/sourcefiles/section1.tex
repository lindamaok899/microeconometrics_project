We review three papers that provide a seamless synergy into our project that estimates the impact of foreign direct investment on firm performance by taking a holistic approach to the literature. We first take a look at a paper by \cite{borin2016foreign} which displays the effects on firm performance in the firm’s country of origin when they decide to invest abroad, thus becoming a multi-national enterprise. In an intermediate stage, we review a paper by Bajgar and Javorcik (2020)  that investigates analyze spillover effects from FDI on domestic firm performance resulting from links along the supply chain. To conclude our review and introduce section 2, we review \cite{chen2011} studies the role of the origin of FDIs on target firm performance changes. \\ \par

\cite{borin2016foreign} investigate the ex-post effects of foreign direct investment (FDI) on firm performance using data from the Bank of Italy’s annual Survey of Industrial and Service Firms (the Invind survey) as well as balance sheet data from the Company Accounts Data Service (henceforth CADS).  They extend the discussion from empirical and theoretical literature that shows how multinational firms (MNE) exhibit a competitive advantage before investing abroad, by conducting microeconomic data analysis to evaluate the policy implications of firm heterogeneity. The authors specify the best-case scenario to be the implementation of policy measures and internationalization strategies that are capable of enhancing both firm performance and employment. The ex-ante causal relationship (from performance to internationalization) in \cite{borin2016foreign}’s paper introduces a severe form of endogeneity, in that ex-post performance might reflect not only foreign investment, but also pre-existent advantages in terms of managerial ability, know-how and technology. To accurately evaluate the ex-post effects of FDI that take into account the inherent self-selection problem, the authors use a propensity score matching procedure to analyze the causal relationship between firm performance and foreign direct investment, looking in particular at the potential gains in terms of productivity and potential losses in terms of employment in the parent firm due to the acquisition of multinational status; as well as to evaluate whether these effects are evenly spread across new MNEs or concentrated among certain groups of investors. To implement the propensity matching estimation, the authors choose a company, ex-ante very similar to the first one, but that does not choose to invest abroad to act as a proxy of the unobservable counterfactual, enabling them to compare the evolution of new MNEs with the performance of the exact same company with no investment; since it is not possible to observe the same company in these two different scenarios. They find that firms investing abroad for the very first time, especially in advanced economies, show higher productivity and employment dynamics in the years following the investment: the average positive effect on TFP is driven by new multinationals operating in specialized and high-tech sectors, while the positive employment gains are explained by an increase of the white collar component. On average, the authors do not find a negative effects on the parent firm's blue collar component. \\ \par

\cite{bajgar2020climbing} analyze spillover effects from FDI on domestic firm performance resulting from links along the supply chain. More specifically, they investigate the relationship between quality upgrading by Romanian exporters and the presence of foreign affiliates in upstream and downstream industries, respectively. They use customs data merged with firm-level data for the years 2005 to 2011 and find a positive and robust link between the product quality of domestic exporters – proxied by two measures established in the literature, namely unit values and a measure proposed by \cite{khandelwal2013trade} – and the presence of foreign-owned firms in upstream (i.e. input-supplying) industries. The results indicate that a one percentage point increase in FDI presence upstream is related to a roughly 0.5\% increase in product quality of domestic exporters, on average and ceteris paribus. The positive relationship between export quality and the presence of foreign-owned firms in downstream industries is less robust. To estimate these effects, the authors employ OLS regressions with product quality as the dependent variable and a measure of FDI presence in upstream and downstream industries, respectively, as the main independent variable of interest. They control for firm-product-destination fixed effects, region-time fixed effects and linear time trends at the region-industry level. However, the authors acknowledge that reverse causality could be a problem, for example if foreign firms primarily invest in industries where high-quality domestic inputs become increasingly available, and these developments are not captured by the linear time trends. To mitigate these concerns, they lag their FDI variables by one year in all specifications. In addition, they perform a “strict exogeneity” test suggested by \cite{wooldridge2010econometric}, including also contemporaneous and lead FDI values into their model. They argue that if foreign firms enter Romania as a consequence of quality upgrading rather than the other way around, the coefficients on the lead values of FDI should also be statistically significant, which is not the case. They conclude that the increase in foreign presence in the upstream industries over the studied period corresponded to a circa 4\% increase in the quality of exports by local firms. \\ \par

\cite{chen2011} studies the role of the origin of FDIs on target firm performance changes. It uses data on acquisitions of U.S firms between 1979 and 2006, comparing firm level performance indicators before and after acquisition, focusing on the difference between firms acquired by domestic firms (USFs), firms from industrialized countries (ICFs) and firms from developing countries (DCFs). The identification strategy is based on a diff-in-diff analysis which gives rise to a twofold selection problem. The empirical set up defines a control group which is domestically acquired firms and two treatment groups. As trans-boundary firm acquisition is more challenging than domestic acquisitions (Helpman et al 2004), firms that engage in the former are likely to be different from those that invest domestically, which might translate into "more skillful" target firm selection on part of foreign acquirers. Hence, the firms chosen for foreign investments are likely to structurally differ pre-treatment from those firms that are chosen for domestic acquisitions. Secondly, selection criteria for target firms might also differ based on the origin of the acquiring firm. The author argues that systematic differences can be expected in the aimed at restructuring process depending on the origin of the acquirer, due to structural differences between these groups in technological progress and relative input costs between target and acquirer. Consequently, selection bias might be found in the comparison of control and the treatment groups, as well as in the comparison of the two treatment groups. To solve this issue the author implements a propensity score matching procedure. The scoring is achieved through a multinomial logit model estimation of observables available to acquires before the acquisition that provide information on present and potential future target firm performance, as well as year, industry and state fixed effects. The matching is done using a kernel matching procedure. This step allows to impose pre-treatment homogeneity in firm performance across comparison groups and thus to interpret any difference in performance between matched pairs to result from the difference in treatment. The results of this estimation show that firms acquired by ICFs show 13\% higher improvements in terms of labour productivity, a 10\% higher profit increase and a 19\% higher increase in sales in the five years post acquisition relative to the year before the acquisition, compared to firms acquired by USFs. Between those two groups no difference in overall employment changes was found. For firms acquired by DCFs the author finds that labour productivity gains and changes in sales and employment are 1\% lower than those of domestic acquisitions, albeit higher profit gains. Comparing firm-level effects of FDIs originating from industrialized countries to those from developing countries, gains in sales, labour productivity and employment are significantly higher for target firms acquired by ICFs. \\ \par

In Section $2$, we will provide a discussion of the most important features of the data we will be working with, including some interesting patterns and correlations that we have discovered in the data. We also provide some summary statistics and graphs of the variables of interest.
