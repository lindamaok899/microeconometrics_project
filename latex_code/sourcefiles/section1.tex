Evaluating the impact of foreign direct investment (FDI) on firm-level performance has received a lot of attention in the literature. Researchers are interested in the impact of FDI on the investor (or parent) firm and the target firm of FDI, as well as in potential spillover effects. We briefly review three papers representing these three strands of investigation, focusing on their empirical strategies. \\ \par

\cite{borin2016foreign} investigate the effects of investing abroad on the performance of the parent firm using survey and balance sheet data for Italy. The authors are confronted with a severe form of endogeneity as firms self-select into investing abroad: firms investing abroad might be inherently different in terms of managerial ability, know-how, and technology, affecting their performance before and after the investment. In light of this self-selection problem, the authors use a propensity score matching procedure to analyze the causal relationship between firm performance and FDI, looking in particular at the effects on productivity and employment of the parent firm. They also evaluate whether these effects are the same across parent firms or concentrated among certain groups of investors. They find that firms investing abroad for the first time show higher productivity and employment dynamics in the years following the investment. \\ \par

\cite{bajgar2020climbing} analyze spillover effects from FDI on domestic firm performance resulting from links along the supply chain. More specifically, they investigate the relationship between quality upgrading by Romanian exporters and the presence of foreign affiliates in upstream and downstream industries, using customs data merged with firm-level data. They employ OLS regressions with product quality as the dependent variable and a measure of FDI presence as the independent variable of interest. They control for firm-product-destination fixed effects, region-time fixed effects and linear time trends at the region-industry level. To alleviate the concern of reverse causality, they lag their FDI variable by one year. The results show that a one percentage point increase in FDI presence upstream is related to a 0.5\% increase in product quality of domestic exporters, on average and ceteris paribus. \\ \par

\cite{chen2011} studies the role of the origin of FDIs on target firm performance changes using data on acquisitions of US firms. The authors differentiate between firms acquired by domestic firms (USFs), firms from industrialized countries (ICFs), and firms from developing countries (DCFs). The identification strategy is based on a diff-in-diff analysis combined with propensity score matching. The scoring is achieved through a multinomial logit model estimation of observables available to acquires before the acquisition that provide information on present and potential future target firm performance, as well as year, industry and state fixed effects. The matching is done using a kernel matching procedure. This allows to impose pre-treatment homogeneity in firm performance across comparison groups and thus to interpret any difference in performance between matched pairs to result from the difference in treatment. The results show that gains in sales, labor productivity, and employment are highest for firms being acquired by ICFs.
