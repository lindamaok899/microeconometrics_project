When analyzing the impact of FDI on wages and employment at the firm level, the  motive of the foreign investor might also play a role. For example, investors trying to sell their products in the destination country of their investment  may be particularly concerned about their reputation in the host country, potentially leading to higher wage payments. Our dataset also provides information on the type of FDI and distinguishes across four categories: exports-oriented FDI (1),  technology-intensive FDI (2), and domestic market seeking FDI (3). Of the firms receiving FDI, 44\% receive domestic-market seeking FDI, follow by 35\% that receive technology-intensive FDI. The share of firms receiving export-oriented FDI is 21\%. \\ \par

Figure \ref{fig_treatment_type} provides a graphical illustration of the main variables by type of FDI, displaying the means of wages, employment, TFP, export intesnity and R\&D for 2015 and 2017, respectivly. In addition, the graph displays the means for firms not receiving FDI (0). The corresponding Tables can be found in the Appendix (Tables \ref{sumstat_treatment_type_pre} and \ref{sumstat_treatment_type_post}). While the differences in means across types of FDI are relativley small when it comes to wages, employment, export intensity and TFP both in 2015 and 2017, the differences across FDI types are large regarding R\&D activities.
